# Precision of Fold Change Estimates and Benchmarking

## DESeq2 Approach and Comparison

- **Overview of DESeq2**:
  - Utilizes empirical priors to achieve shrinkage of Log Fold Change (LFC) estimates.
  - Goal: Enhance stability and interpretability of results for RNA-seq data analysis.

- **Benchmarked Against**:
  - **GFOLD**:
    - Capable of analyzing experiments without replication.
    - Handles experiments with and without replicates.
  - **edgeR**:
    - Uses pseudocount-based shrinkage termed predictive LFCs.

## Results

- **Error Metrics**:
  - DESeq2 consistently shows low root-mean-square error and mean absolute error.
  - Performance held across various sample sizes and LFC distribution models.

- **Performance Specifics**:
  - **GFOLD**:
    - Comparable error to DESeq2 overall.
    - Performs worse for larger samples when focusing on differentially expressed genes.
  - **edgeR**:
    - Similar performance to DESeq2 for differentially expressed genes.
    - Higher error across all genes compared to DESeq2.

## Simulation and Practical Insights

- **Simulations**:
  - Simulations validate theoretical behavior of algorithms.
  - Important limitation: Simulations do not reflect real-world complexities precisely.

- **Practical Considerations**:
  - RNA-seq data brings complications due to unknown underlying truth.
  - Real-world benchmarks are necessary for practical validation.

---

### Example Visual (for Presentation Slide)

```latex
\begin{frame}{Precision of Fold Change Estimates}
    \frametitle{Benchmarking DESeq2}
    \begin{itemize}
        \item \textbf{DESeq2} 
        \begin{itemize}
            \item Uses empirical priors for shrinkage of LFC estimates.
            \item Aims to improve stability and interpretability of RNA-seq data results.
        \end{itemize}
        \item Compared against \textbf{GFOLD} and \textbf{edgeR}
        \begin{itemize}
            \item GFOLD: Handles non-replicate data.
            \item edgeR: Provides predictive LFCs using pseudocount.
        \end{itemize}
        \item \textbf{Results \& Metrics}
        \begin{itemize}
            \item DESeq2 shows consistently low error rates.
            \item GFOLD and edgeR have specific strengths but higher overall errors.
        \end{itemize}
    \end{itemize}
\end{frame}
```

---

## Conclusion

- DESeq2 demonstrates robust performance for fold change estimation in RNA-seq data.
- Benchmarking against GFOLD and edgeR shows its advantage in terms of error metrics across sample sizes.
- Simulations are valuable but require real-data benchmarks to validate practical utility.

\[ \text{Source: Love\_DESeq2.pdf} \]