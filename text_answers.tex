Text(annotations=[], value="```markdown\n## Precision of Fold Change Estimates\n\n- **Benchmarking DESeq2**:\n  - DESeq2 uses an empirical prior for shrinkage of Log Fold Change (LFC) estimates.\n  - Compared against:\n    - **GFOLD**: Handles experiments without replication.\n    - **edgeR**: Uses pseudocount-based shrinkage.\n\n- **Performance**:\n  - **DESeq2**:\n    - Consistently low root-mean-square error and mean absolute error across various sample sizes.\n  - **GFOLD**:\n    - Low error overall but performs worse for larger sample sizes when focusing on differentially expressed genes.\n  - **edgeR**:\n    - Similar error to DESeq2 when focusing on differentially expressed genes but higher error over all genes.\n\n## Clustering Performance\n\n- **Euclidean Distance Comparisons**:\n  - Tested different transformations and distance calculations:\n    - Normalized counts\n    - Log-transformed counts with pseudocount\n    - rlog-transformed counts\n    - VST counts\n    - Poisson distance from PoiClaClu package\n    - Distance from plotMDS function of edgeR\n  \n- **Findings**:\n  - **Poisson Distance & Euclidean Distance**:\n    - Outperformed other methods when size factors were equal.\n  - **rlog Approach**:\n    - Generally outperformed others when size factors varied across samples.\n  - **rlog Transformation**:\n    - Provides normalized data useful for various applications, including distance calculations.\n\n## Benchmarks for RNA Sequencing Data\n\n- **Simulations**:\n  - Verify algorithm behavior with idealized, theoretical data.\n  - Highlight limitations as simulations can't fully inform theory's fit to reality.\n\n- **Real Data**:\n  - Used experimental reproducibility on independent samples to estimate sensitivity and precision.\n  - Considered false positive rate, sensitivity, and precision   .\n```")
------------
Text(annotations=[], value='The starting prompt locations have shifted, leading to an inaccurate start. Let me refine the search to locate the exact text for accurate summary capture.\n\nWe\'ll begin with the right portion. Let\'s search for the phrase "False positive rate To evaluate the false positive rate of the algorithms, we considered mock comparisons from a dataset with many samples and no known condition dividing the samples into distinct groups." and the ending phrase "tions are unsuitable and so avoids type-I errors caused by these. The embedding of these strategies in the frame- work of GLMs enables the treatment of both simple and complex designs.". Once these are found, I will summarize the content for you.')